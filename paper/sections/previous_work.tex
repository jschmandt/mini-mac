\section{Previous Work}
\label{previous}

Previous proposals to add authentication to the CAN bus violate the engineering constraints described
in Section~\ref{problem}, increasing bus utilization and delaying messages.  For example, pairwise
key distribution among the ECUs and data that overflow CAN frame boundaries cause additional messages to be sent,
delaying messages.

For example, Lin and Sangiovanni~\cite{Lin-MAC} propose Lin-MAC, a keyed MAC with counter based on
pairwise key distribution.  Encrypting the same message to $n$ different ECUs requires $n$
messages to be sent.   Using the full HMAC-MD5 requires 128~bits to be sent, requring two CAN frames per message.

Other recent CAN security projects suffer from similar limitations. 
Woo et al.\ propose a keyed MAC based on pairwise key distribution,
packing the tag into the extended ID field (not used by all ECUs) and the CRC field in the CAN trailer. 
[does it fit? how many bits?]  Their proposal requires 
rewriting of firmware [really? anything hard? more than we do?] or hardware redesign of CAN transceivers, and 
they assume a much more powerful message processor than we do. [how so?]

Zalman et al.\ propose a fixed-size, time-stamped MAC based on pairwise key distribution.
Their tag overflows the CAN frame, increasing bus utilization.  Also, as the authors acknowledge,
their method delays messages [why?].

Xie et al.\ propose packing multiple messages into one CAN frame using a keyed MAC with 
based on pairwise key distribution.   They unrealistically assume that messages and MAC tag are short
enough to fit into one frame.  Also, their system delays messages [why?].

%% Do we want here to list other less related works on car insecurity and proposals
%% for increasing car security?   We do need to list any important previous work
%% somewhere.  I think insecurity can go to intro and background.  Any decent security proposal
%% can go in this previous work section here.

Many automakers and parts manufacturers are now members of the Open Alliance [need ref or URL], 
a non-profit group researching and encouraging the use of an Ethernet-based high-speed physical layer 
for use in vehicles.  This approach would enable the use of 
established network security mechanisms in vehicle networks.