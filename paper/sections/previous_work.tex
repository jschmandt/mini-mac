\section{Previous Work}
\label{previous}

Previous proposals to add authentication to the CAN bus violate the engineering constraints described
in Section~\ref{problem}, increasing bus traffic and delaying messages.  For example, pairwise
key distribution among the ECUs and data that overflow CAN frame boundaries cause additional messages to be sent,
delaying messages.

For example, Lin and Sangiovanni~\cite{Lin-MAC} propose Lin-MAC, a keyed MAC with counter based on
pairwise key distribution.  Encrypting the same message to $n$ different ECUs requires $n$
messages to be sent.   Using the full HMAC-MD5 requires 128~bits to be sent per message, requiring two CAN frames per message.

Other recent CAN security projects suffer from similar limitations. 
Woo et al.~\cite{Woo-14} propose a keyed MAC based on pairwise key distribution,
packing the tag into the extended ID field (not used by all ECUs) and the CRC field in the CAN trailer
(for a related proposal of ours, see Section~\ref{addingbits}). 
These bits fit only if the ECUs use the extended ID field. 
Their proposal requires a hardware redesign of CAN transceivers or rewriting a layer of message transmission firmware.
Care should be taken when comparing their computation times because
they assume a much more powerful message processor than we do. 

Zalman and Mayer~\cite{Zalman-14} propose a fixed-size, time-stamped MAC based on pairwise key distribution.
Their tag overflows the CAN frame, increasing bus traffic, and as the authors acknowledge,
delaying messages.

Xie et al.~\cite{Xie-15} propose packing multiple messages into one CAN frame using a keyed MAC with 
pairwise key distribution.   They unrealistically assume that the messages and MAC tag are short
enough to fit into one frame.  Also, by queuing messages into batches, their system delays messages.

%redacted 

Schmandt first worked on the problem in spring 2015
as a research project for Sherman's CMSC-652 cryptology class at UMBC.
Unknown to us until recently, independently, Bravo et al.~\cite{MIT2015} carried out a similar effort 
for Rivest's 6.857 computer and network security class at MIT.  Both projects use HMACs, but in different ways.
The MIT project, assuming lossless channels, initializes each channel with a precomputed HMAC chain.
Using a software CAN bus simulator, they show that their scheme increases message latency by
approximately a factor of three, and increases bus traffic.  By contrast, our method does not increase
bus traffic and does not increase message latency, and our use of message history protects against transient
attackers who know the keys.

%[do we want to cite any other works on improving car security? If so, do it here.]

Many automakers and parts manufacturers are now members of the 
Open Alliance,\footnote{http://www.opensig.org}
a non-profit group researching and encouraging the use of an Ethernet-based high-speed physical layer 
for use in vehicles.  This approach would enable the use of 
established network security mechanisms in vehicle networks.


