\section{Discussion}
\label{discuss}

In this section we discuss how to resynchronize ECUs, 
how to lengthen the mini-MAC tag, 
and we list some open problems.

%a radical design alternative, 

\subsection{Resynchronization}
\label{resynch}

Our mini-MAC proposal requires the ECUs to have synchronized counters and message-history states.
Therefore, a mechanism is needed to resynchronize the ECUs in case they ever lose synchronization,
as might happen, for example, by a fault in the ECU or a disruption in message transmission.
[how do ECUs know that they are out of synch?]

Two common solutions are to reset the state to a specified initial state, or for one ECU to select
a new state and communicate that state to the other ECUs (encrypted by a shared secondary communication
encryption key).  [do you have a ref?]

Instead, for enhanced security we propose that each ECU periodically save its state in persistent memory.  
In the initial attempt to resynchronize, each ECU loads its most recent state.  If that fails, then the aforementioned
mechanisms could be applied.

%% this section is rather weak--nothing new.  No anlysis of proposal.
%% It is good to mention need for resynch.  We should not claim any contribution here.

\subsection{Lengthening the Mini-MAC Tag}
\label{addingbits}

It is possible to lengthen the Mini-MAC tag by 
using the two bytes of space allocated for the CRC field in the CAM frame (see Figure~\ref{fig-frame}),
as suggested by Woo et al.~\cite{Woo-14} in a related proposal.
Because a MAC detects transmission errors (in fact, better than a simpler CRC), there is no need for
a CRC in addition to a MAC.  

Increasing the tag length greatly increases the time required for an adversary to forge a valid tag by
finding a collision in the Mini-MAC by exhaustive search.  Table~\ref{tab-taglength} gives the
time to find a Mini-MAC collision for various tag lengths. [under what assuptions?? algorithm, machine, etc?]
For example, increasing the tag from 32 to 48 bits increases this time from
approximately 27.3~minutes to over four days.

%% as we will point out in the security section, exchaustive search attacks to finds collisions are
%% not so useful because there is no way for the adversary to check candidate values.
%% perhaps the best attack is just to inject many messages, hoping one will work, an attack
%% which could be easily detected

To implement this strategy one could modify the lower-level code in the CAN network stack, 
either to perform the MAC calculation there 
or to open the CRC field to the application level to calculate the MAC.

%% compare with Woo (see previous work).
%% we are not first to suggest.  How do we differ?

	\begin{table}	
	\centering
	\caption{Time to find collision for various tag lengths [explain]}\label{tab-taglength}
	\vspace{8pt}
	\begin{tabular}{c|r}%
	\bfseries Tag Length (b) & \bfseries Time to Find Collision\\\hline \csvreader[late after line=\\]%
		{tables/maclen_bfttb.csv}{mac_len=\mac_len,ttb=\ttb}% 
		{\mac_len & \ttb}%
	\end{tabular}
	\end{table}
	
	%% I don't know this mechanism. need to add space between number and unit in each line
	%% headers should be done in latex directly

%\subsection{Design Alternative}
%\label{alternatives}

%% would be nice to discuss our enigneering decisions and alternatives, but we don't have anything interesting to say

\subsection{Open Problems}
\label{open}

Our engineering decisions are driven by a desire to improve vehicular security by adding authentication
to the CAN bus, without increasing bus traffic or delaying messages, and without making any
disruptive changes.  The egregious state of vehicular security, however, demands a radical disruptive
redesign of vehicular computer networks carried out including security as a foundational design
requirement. [refs?]

Design ideas for a replacement network to the CAN bus include the following:  
(1)~Use a well-established high-speed network (such as 802.3 Ethernet) on which
standard security mechanisms (such as IPsec) can be deployed.
(2)~Segegrate nodes on the bus into task-defined groups.
(3)~Protect access to the bus by physically separating critcal and non-critical systems.
In particular, it should not be possible for maleware or faults in entertainment or 
Bluetooth systems to affect braking, steering, or acceleration.  

A separate related problem is to detect vehicular network intrusions. [ref?]  A challenge of such work is that
there is no good reponse of what to do if an intrusion is detected other than to shut down the vehicle safely.

The Car-to-X network~\cite{C2X} is an emerging interconnected collection of vehicles, buildings, signs, and road infrastructure 
to reduce congestion and enable more efficient traffic control. Cars of the future will have to be able to communicate
securely with objects on such networks, requiring authentication and key management beyond Mini-MAC.

%% I don't think the draft question about padding or security of truncated MACs is in a state worthy of mention here.

