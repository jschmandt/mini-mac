\section{Discussion}
\label{discuss}

In this section we discuss how to resynchronize ECUs, 
how to lengthen the mini-MAC tag, 
a radical design alternative, 
and open problems.

\subsection{Resynchronization}
\label{resynch}

Our mini-MAC proposal requires the ECUs to have synchronized counters and message-history state.
Therefore, a mechanism is needed to resynchronize the ECUs in case they ever lose synchronization,
as might happen, for example, by a fault in the ECU or message transmission.
[how do ECUs know that they are out of synch?]

Two common solutions are to reset the state to a specified initial state, or for one ECU to select
a new state and communicate that state to the other ECUs (encrypted by a shared secondary communication
encryption key).  [do you have a ref?]

Instead, we propose that each ECU periodically save its state in persistent memory.  In the initial attempt
to resynchronize, each ECU loads its most recent state.  If that fails, then the aforementioned
mechanisms could be applied.

%% this section is rather weak--nothing new.  No anlysis of proposal.
%% It is good to mention need for resynch.  We should not claim any contribution here.

\subsection{Lengthening the Mini-MAC Tag}
\label{addingbits}

It is possible to lengthen the Mini-MAC tag by 
using the two bytes of space allocated for the CRC field in the CAM frame (see Figure~\ref{fig-frame}),
as suggested by Woo et al.~\cite{Woo-14}.
Because a MAC detects transmission errors (in fact, better than a simpler CRC), there is no need for
a CRC in addition to a MAC.  

Increasing the tag length greatly increases the time required for an adversary to forge a valid tag by
finding a collision in the Mini-MAC by exhaustive search.  Table~\ref{tab-taglength} gives the
time to find a Mini-MAC collision for various tag lengths. [under what assuptions?? algorithm, machine, etc?]
For example, increasing the tag from 32 to 48 bits increases this time from
approximately 27.3~minutes to over four days.

%% as we will point out in the security section, exchaustive search attacks to finds collisions are
%% not so useful because there is no way for the adversary to check candidate values.
%% perhaps the best attack is just to inject many messages, hoping one will work, an attack
%% which could be easily detected

To implement this strategy one could modify the lower-level code in the CAN network stack, 
either to perform the MAC calculation there 
or to open the CRC field to the application level to calculate the MAC.

%% compare with Woo (see previous work).
%% we are not first to suggest.  How do we differ?

	\begin{table}	
	\centering
	\caption{Time to find collision for various tag lengths}\label{tab-taglength}
	\vspace{8pt}
	\begin{tabular}{|c|r|}\hline%
	\bfseries Tag Length (b) & \bfseries Time to Find Collision\\\hline \csvreader[late after line=\\]%
		{tables/maclen_bfttb.csv}{mac_len=\mac_len,ttb=\ttb}% 
		{\mac_len & \ttb}%
		\hline
	\end{tabular}
	\end{table}
	
	%% I don't know this mechanism. need to add space between number and unit in each line
	%% headers should be done in latex directly

\subsection{Design Alternative}
\label{alternatives}

The capabilities of the CAN bus and the ECUs on it are limiting factors in security. There are two major design changes to automotive computer networks that would significantly increase security: 1) Replace CAN with high-speed, well defined network stack elements such as 802.3 Ethernet and IP 2) Segregate nodes on the CAN bus into task-defined groups.

As info-tainment becomes a major task of the modern automotive computer network, the networks used in vehciles must be able to support not only control information but high rate audio and video content. The same type of network can be used for ECU-to-ECU communication, and the Ethernet/IP stack found in many home networks is a natural fit for this task. These networks can utilize IPsec, which can handle data encryption and authentication.

That a car radio could potentially send a message to a brake control unit is a tremendous oversight in design. Although some vehicles do a better job of segregating vehicle control units from entertainment or environment control units than others, there is no standardization and these design decisions seem to stem more from spatial arrangment or bus availablilty than they do for any security-conscious reason. The solution should be to place high responsibility nodes (ECUs that control, for example, brakes and acceleration) on physically separate networks from low-responsibility nodes (the radio, blinkers, etc) and nodes designed to communicate to the Car-to-X network (navigation systems, in-car Internet).

\subsection{Open Problems}
\label{open}

A relevant question to this work is the effectiveness of hash functions over small inputs. As previously noted, the messages on the CAN bus are 8 B, and before any practical use of a solution like Mini-MAC could be made, this question must be answered to a satisfying degree. Many hash implementations require padding of inputs to fit fixed size inputs, but what ratio of padding to message is acceptable before the hash loses effectiveness? These are not questions unique to Mini-MAC, but it is a protocol which could be severely compromised from a hash which is weak under heavy padding as the message is so small compared to the input block size. \textcolor{red}{Note: You marked this paragraph unclear -- could you elaborate?}

An important open problem is that of securing remote bus access. Auto manufacturers do not publicly acknowledge the issues present in modern automobiles, but do participate in groups such as the Open Alliance designed to change the paradigm of vehicle computer networks. As long as networked nodes such as the entertainment system are connected to both CAN nodes and the larger Internet there will be vulnerabilities allowing attackers to remotely gain access to the CAN bus.

A separate but related problem in vehicle security is the detection of network intrusions. This work deals with the prevention of attacks on the CAN bus, whereas detection mechanisms serve to alert the driver when an attack is taking place, in the event that protection mechanisms fail. The problem with current attack detection work is that upon detection, there is no procedure to mitigate and recover from the attack apart from stopping the vehicle (or similar real-time system).

The Car-to-X network is an emerging interconnected collection of vehicles, buildings, signs, and road infrastructure 
to reduce congestion and enable more efficient traffic control~\cite{C2X}. Cars of the future will have to be able to communicate
securely with objects on such networks, requiring authentication and key management beyond mini-MAC.

%Lin-MAC claims to implement a suitable key distribution infrastructure and counter resynchronization mechanism that is suitable for implementation on ECUs on the CAN bus. Before any system becomes practically implementable, these claims must be verified.