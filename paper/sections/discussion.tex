\section{Discussion}
\label{discuss}

In this section we discuss how to resynchronize ECUs, 
how to lengthen the mini-MAC tag as an optional improvement, 
and we list some open problems.

%a radical design alternative, 

\subsection{Resynchronization}
\label{resynch}

Our mini-MAC proposal requires the ECUs to have synchronized counters and synchronized message-history states.
Therefore, a mechanism is needed to resynchronize the ECUs in case they ever lose synchronization,
as might happen, for example, by a fault in the ECU or a disruption in message transmission.

Two common solutions are to reset the state to a specified initial state, or for one ECU to select
a new state and communicate that state to the other ECUs (encrypted by a shared secondary communication
encryption key).  

Instead, for enhanced security we propose that each ECU periodically save its state in persistent memory.  
In the initial attempt to resynchronize, each ECU loads its most recent state.  If that fails, then the aforementioned
mechanisms could be applied.  Section~\ref{arch} explains how message history helps guard against
replay attacks upon resynchronization.

A limitation of the CAN bus is that it provides no mechanism for detecting when ECUs are out
of synchronization.  In some cases, by monitoring observable conditions on the bus, 
an ECU might detect that another ECU is not responding to a message, which might be
the result of a synchronization failure.  A tradeoff in our design is that, by using
counters and message history to authenticate messages, 
mini-MAC increases the opportunities for possible synchronization failures,
and this opportunity increases with larger values of~$\lambda$.

We do not know the typical frequency of synchronization failures,
and we did not observe any.

\subsection{Lengthening the Mini-MAC Tag}
\label{addingbits}

Optionally, it is possible to lengthen the Mini-MAC tag by 
using the two bytes of space allocated for the CRC field in the CAM frame (see Figure~\ref{fig-frame}),
as suggested by Woo et al.~\cite{Woo-14} in a related proposal.
Because a MAC detects transmission errors (in fact, better than does a simpler CRC), there is no need for
a CRC in addition to a MAC.  

Increasing the tag length greatly increases the time required for an adversary to forge a valid tag by
finding a valid Mini-MAC tag by exhaustive search.  
Table~\ref{tab-taglength} gives the expected time for an adversary to send a forged message 
that will be accepted by another ECU, for various tag lengths.  
For example, increasing the tag from 8 to 32~bits increases this time from
approximately 3.2~seconds to over 621~days.

%Here, the main limiting assumption is 
%the frequency with which messages can be sent on the CAN bus. 

To implement this strategy one could modify the lower-level code in the CAN network stack, 
either to perform the MAC calculation there 
or to open the CRC field to the application level to calculate the MAC.

\subsection{Open Problems}
\label{open}

It would be interesting to implement and test Mini-MAC in an actual vehicle.

Our engineering decisions are driven by a desire to improve vehicular security by adding authentication
to the CAN bus, without increasing bus traffic or delaying messages, and without making any
disruptive changes.  The egregious state of vehicular security, however, demands a radical disruptive
redesign of vehicular computer networks carried out including security as a foundational design
requirement~\cite{Wolf2006}. 

Design ideas for a replacement network to the CAN bus include the following:  
(1)~Use a well-established high-speed network (such as 802.3 Ethernet) on which
standard security mechanisms (such as IPsec) can be deployed.
(2)~Segregate nodes on the bus into task-defined groups.
(3)~Protect access to the bus by physically separating critical and non-critical systems.
In particular, it should not be physically possible for malware or faults in entertainment or 
Bluetooth systems to affect braking, steering, or acceleration. 
(4)~Physically protect the ECUs. 

% REDACTED
Additional ideas, suggested by Phatak,\footnote{Private correspondence with Dhananjay Phatak.} 
include:
(5)~Physically secure the bus to prevent any access, except through
a read-only diagnostic port.
(6)~At pseudorandom time intervals, pseudorandomly change the ECU ID numbers, 
based on a key shared by all ECUs used only for this purpose.

A separate related problem is to detect vehicular network intrusions~\cite{Otsuka2014}. A challenge of such work is that
there is no good response of what to do if an intrusion is detected other than to shut down the vehicle safely.

The Car-to-X network~\cite{C2X} is an emerging interconnected collection of vehicles, buildings, signs, and road infrastructure 
to reduce congestion and enable more efficient traffic control. Cars of the future will have to be able to communicate
securely with objects on such networks, requiring authentication and key management beyond Mini-MAC.

