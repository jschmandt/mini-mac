\section{Adversarial Model}
\label{adversary}

We consider three classes of adversaries:

\begin{description}

	\item [Type 1]
	(\textit{Strongest adversary}): A permanent entity on the CAN bus with a valid key for the MAC it wishes to generate.
	
	\item [Type 2]
	(\textit{Strong adversary}): A permanent entity on the CAN bus without a valid key for the MAC it wishes to generate.
	
	\item [Type 3]
	(\textit{Weak adversary}): A transient entity on the CAN bus without a valid key for the MAC it wishes to generate.
	
\end{description}

For example, a Type~1 adversary might be a compromised ECU on the CAN bus.  A Type~2 adversary might be a malicious piece of hardware attached
to the CAN bus.  A Type~3 adversary might be a criminal who has gained temporary access to the CAN bus, perhaps via a wireless channel from
another nearby car.  For a Type~3 attacker, we assume the adversary's access to the CAN bus is limited to minutes, not hours.  
The differences among these attackers are what keys they know and for how long they have access to the CAN bus.
This project aims to defend against Type~2 and Type~3 adversaries.   Our techniques do not protect against 
a Type~1 adversary, who can spoof any message to any member of any group to which he holds the key. 

Motivation of the attacker includes criminal mischief (e.g., crashing car, destroying property) and theft.  
For example, spoofing messages on the CAN bus can unlock doors, disable brakes, and accelerate the car.
Goals of the attacker include spoofing or replaying messages on the CAN bus that will be accepted by an ECU
as valid.

We assume the attacker posses complete knowledge of the CAN bus and ECUs, including all protocols, message IDs, and formats.
We assume the attacker has substantial computing power, reliable access to the CAN bus, and is able to 
monitor and inject messages on the bus.  We assume, however, that the adversary cannot inject more than 40 messages
per second without detection.

We assume the ECUs are trustworthy and that the adversary cannot break standard cryptographic functions 
including encryption and hash functions.

This work does not address Denial-of-Service (DOS) attacks aimed at preventing a driver from using her vehicle.  
For example, our techniques do not prevent an attacker from flooding the CAN bus with messages.  There are many
simple physical DOS attacks, such as slashing a tire, cutting wiring, or draining fuel, yet
a DOS attack on the CAN bus while the vehicle is operating could be dangerous 
(e.g., losing control on a high-speed turn).  Guarding against DOS attacks is difficult and would require fundamental
changes to the CAN bus.\footnote{Another potential DOS attack, which more generally attacks
all electronics in a vehicle, is to radiate the vehicle with an electro-magnetic pulse.
Doing so might be useful to law enforcement in extreme circumstances to stop certain dangerous high-speed criminals, 
if the pulse could be directed in a controlled, focused, and well-targeted manner.}

Although we assume the adversary has complete knowledge of the target technology, in practice the adversary must
deal with the straightforward yet cumbersome task of learning this technology, which may include new ECUs and message formats.


