\section{Conclusion}

We propose Mini-MAC, the first variable-length MAC protocol for the CAN bus that adds no bus traffic overhead, 
allowing it to be used in vehicular systems with time-sensitive messages.  The truncated HMAC protects against
message injection by adversaries who do not know the ECU keys.  The counter and message history protect
against replay attacks.  

Limited message size, the need not to delay messages, the limited computational power of the ECUs,
and the relative ease of gaining access to the bus 
severely restrict how well the CAN bus can be protected.  Mini-MAC meaningfully raises the bar on vehicular security,
approaching (we conjecture) the limits of what is possible for authentication strength in this highly
constrained environment.

