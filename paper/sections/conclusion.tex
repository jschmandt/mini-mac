\section{Conclusion}
\label{conclude}

We propose Mini-MAC, the first variable-length message authentication
protocol for the CAN bus that adds no bus traffic overhead, 
allowing it to be used in vehicular systems with time-sensitive messages.  
The truncated keyed HMAC protects against
message injection by adversaries who do not know the ECU keys.  
The counter and message history protect against replay attacks.  
Message history also protects against all transient attackers, even if they know the authentication keys.

Limited message size, the need not to delay messages, the limited computational power of the ECUs,
and the relative ease of gaining access to the bus severely restrict how well the CAN bus can be protected.  
Mini-MAC meaningfully raises the bar on vehicular security,
approaching (we conjecture) the limits of what is possible for authentication strength in this highly
constrained environment.

Objects and systems in the emerging Internet of Things and Car-to-X network
will face similar authentication challenges to those faced by 
vehicular ECUs.  We hope, for example, that the designers of toaster ovens will not
separately create ad-hoc security mechanisms for networked communications with
the refrigerator, home electrical system, home owner, and toaster manufacturer.
Instead, networked objects of the future need to be built with 
control units that can execute resilient general-purpose standard cryptographic 
functions and protocols suitable for the Internet of Things.  We hope that some
of the ideas from Mini-MAC will be helpful toward that objective.

