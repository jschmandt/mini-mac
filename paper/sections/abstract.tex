\begin{abstract}
We propose Mini-MAC, a new message authentication protocol that works in existing automotive computer networks without adding any bus overhead. 

The CAN bus is a low-speed network between electronic control units (ECUs). It is extremely vulnerable to malicious actors with bus access. Traditionally, Message Authentication Codes (MACs) verify the identity of the sender of a message, and variants prevent message replay attacks; however, standard or time-seeded MACs are unsuitable for use on the CAN bus because of short data payload size restrictions. Our work with a smaller footprint alternative to the traditional MAC raises the bar of vehicle network security. Mini-MAC causes no increase in bus traffic, and incurs a very small performance penalty relative to the provably secure HMAC. It is the first work to combine these two tenets for vehicle computer networks. Mini-MAC is based on a time-seeded HMAC augmented with message history and truncated to fit available message space. Mini-MAC is not suitable for long-term defense against dedicated attackers, but rather is designed to protect the physical safety of drivers against opportunistic attackers in the proposed Car-to-X environment.

\end{abstract}