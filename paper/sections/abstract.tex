\begin{abstract}
We propose Mini-MAC, a new message authentication protocol that is designed to work in existing automotive computer networks without adding any bus overhead. 

The CAN bus is a low speed network between electronic control units (ECUs). It is extremely vulnerable to malicious actors with bus access. Traditionally, Message Authentication Codes (MACs) are used to verify the identity of the sender of a message, and variants are used to prevent message replay attacks; however, standard or time-seeded MACs are unsuitable for use on the CAN bus because of data payload size restrictions. Our work presents a smaller footprint alternative to the traditional MAC which aims to raise the bar of automotive network security. The key elements in Mini-MAC are 1) It causes no increase in bus traffic 2) It incurs a very small performance penalty relative to HMAC. It is the first work to combine these tenets for automotive computer networks. Mini-MAC is based on a time-seeded HMAC, and is augmented with message history and then truncated to fit available message space. Mini-MAC is not suitable for long-term defense against dedicated attackers, but rather is designed to protect the physical safety of drivers against opportunistic attackers in the Car-to-X environment.

\end{abstract}