\section{Mini-MAC}
\label{mini-mac}

Mini-MAC is a group-keyed lightweight variable-length truncated HMAC that
depends on a counter and on recent message history.  It does not increase
bus traffic or delay messages.  We explain Mini-MAC in three layers
of abstraction: architecture, design, and  implementation.
We also describe an enhanced message history option
and comment on key distribution.

\subsection{Architecture}
\label{arch}

Four core architectural elements characterize Mini-MAC:
(1)~Variable-size output to fit available space in the CAN packet.
(2)~Shared keys among groups of ECUs to avoid increasing bus traffic.
(3)~A counter to mitigate replay attacks.
(4)~Message history to defeat transient attackers, 
to mitigate replay attacks after counter resynchronization, 
and to serve as ``salt'' against possible hypothetical precomputation attacks 
(even though we know of none).
Elements (1) and (2) trade authentication strength for performance;  see
Section~\ref{security} for a discussion of this tradeoff.

To avoid sending long tags using additional messages, Mini-MAC truncates the
HMAC tag to fit available space in the CAN frame 
(typically the resulting truncated tag is approximately four bytes).  

To avoid increasing bus traffic by separately authenticating the same message to different recipients, 
Mini-MAC uses long-term shared group keys instead of pairwise keys.  For example, in Figure~\ref{fig-key}, ECUs
2, 3, and 4 share the group key~$k_3$.
Group key distribution is possible because, at system design, the designer knows which ECUs communicate with which others. 
There is no need for a dynamic group key update capability because group membership will never change.

Using a counter is a simple, standard, and effective way to mitigate replay attacks.  

Each ECU saves recent messages sent on the bus
(only for each key group to which the ECU belongs)
and concatenates them into the HMAC input. 
More specifically, each ECU saves a separate message history for the 
successfully authenticated messages received for each group key.  
Thus, the adversary cannot insert a message into any message history without 
successfully forging a message.

There are three reasons for the message history architectural element:  
First, it potentially adds protection against any transient
attacker (even one who knows the group keys) who cannot deduce enough of the 
message history.  
Second, it adds an additional layer of protection against replay attacks.
This layer is useful if the counters need to be resynchronized (see Section~\ref{resynch}).
Consider what happens if a group of ECUs reset their counters and message history to
a previously stored synchronized state.  Without message history, the network
is potentially vulnerable to a replay attack of messages sent from a time 
the reset counter state was previously used.  With message history, it is extremely
likely that the unfolding message history will rapidly differ.
Third, the history adds an unpredictable ``salt'' that might mitigate some possible
hypothetical precomputation attacks.  Whereas counter values are predicable, message history
is not.

	\begin{figure}
		\centering
		\includegraphics[width=\columnwidth]{figures/key_distribution.png}
		\caption{Mini-MAC key distribution.  Each group of ECUs shares an authentication key.}
		\label{fig-key}
	\end{figure}
	
\subsection{Design}
\label{design}

	\begin{figure}
		\centering
		\includegraphics[width=\columnwidth]{figures/minimac_diagram.png}
		\caption{Mini-MAC. The Mini-MAC tag of a message $M_n$ is a truncated keyed HMAC of 
		the concatenation of $M_n$, a counter~$C$, and the most recent $\lambda$ messages.}
		\label{fig-minimac}
	\end{figure}

Figure~\ref{fig-minimac} shows how to compute 
$\hbox{Mini-MAC}(k,M_n,s,C,\hbox{History})$, 
where $k$ is the group key, 
$M_n$ is the current message, 
$s$ is the number of available bits in the current CAN frame, 
$C$ is the message counter, 
and $\hbox{History} = (M_{n-{\lambda}}, \ldots, M_{n-1})$ 
is the sequence of the most recent
valid $\lambda$ messages for key~$k$.

The Mini-MAC tag is computed as

\begin{equation}
\hbox{trunc(}s,\text{HMAC}(k,\text{Input})),
\end{equation}

\noindent where HMAC is the underlying HMAC and $\hbox{trunc}(s,\cdot)$
extracts $s$ bits from its input.  The input to HMAC is computed as

\begin{equation}
\text{Input} = M_n \concat C \concat (M_{n-{\lambda}} \concat \cdots \concat M_{n-1}) .
\end{equation}

\subsection{Implementation}
\label{implementation}

We implemented Mini-MAC using three different component hash functions (MD5, SHA-1, and SHA-2) to compare the resulting running times.  
Each group key is 128~bits.  The $\hbox{trunc}(s,\cdot)$ function extracts 
the $s$ most significant bits of its input, which is the way recommended by NIST
for truncating a hash tag~\cite{nist107}.

We recommend a 64-bit counter, which (assuming at most 40 message per second) ensures no repeated counter state for 20 years
of continuous operation (32 bits would prevent counter roll-over for 20 years with four hours of driving a day).

In our implementation, we artibrarily used $\lambda = 5$, 
but we recommend a higher value (say, $\lambda = 16$) to reduce the chance of a repeated state.
See Section~\ref{history} for an enhanced message history feature.

It may be necessary to instruct all ECUs upon start-up to delay sending messages 
for a small period of time, to prevent fast-booting devices from sending messages 
before all members of the key group are able to receive the message and maintain synchronization.

For \textbf{HMAC-MD5},
we adapted Peslyak's~\cite{Peslyak} implementation of MD5~\cite{MD5} for the MSP430 platform, producing a 128-bit output. 
Despite known collision attacks on MD5~\cite{Wang-MD5}, we consider
MD5 for Mini-MAC for its very fast speed.  

For \textbf{HMAC-SHA-1},
we adapted Conte's~\cite{Conte-SHA1} SHA-1~\cite{FIPS-180-4} implementation for the MSP430 platform, producing a 160-bit output. 
As for MD5, despite known security vulnerabilities in SHA-1 \cite{Wang-SHA1}, 
we consider SHA1 as a potential candidate.

%\footnote{\url{http://bradconte.com/sha1_c}}

For \textbf{HMAC-SHA-2},
we also adapted Conte's~\cite{Conte-SHA256} SHA-2-256 implementation for the MSP430 platform, producing a 256-bit output. 
A member of the SHA-2 family of hash functions, SHA-2 is still in use and is recommended by NIST as a cryptographic hash function, 
though SHA-3 will soon replace it\cite{FIPS-180-4}. We did not use SHA-3 because we did not find an implementation of it
for the MSP430.  Throughout, we shall refer to SHA-2-256 as SHA-2.

%\footnote{\url{http://bradconte.com/sha256_c}}

\subsection{Enhanced Message History}
\label{history}

An enhancement of the message history feature increases security following resynchronization
(see Section~\ref{resynch}) by ensuring that much older messages are incorporated.

By creating a message history sampling from a long enough (e.g., 30~min) stream of messages, 
no transient attacker will have enough time to observe the entire sample.
Powerfully, this strategy defeats all transient attackers, even if they know the group keys.
For example, this strategy defeats an attacker who learns the group keys by compromising
the key distribution process or by physically probing the ECUs, but who cannot
observe bus traffic for a long time.

There are many possible choices for realizing this strategy.  For example,
one might generate a cryptographic hash chain of all messages ever sent.  Alternatively, for
greater speed, one might use a faster non-cryptographic combining function (e.g., CRC).  
Another option would be to compute a hash chain from a limited sample of the stream.

Given the limited power of the ECUs, 
we recommend that the message history be defined in two parts from the recent stream of valid messages
received under the given group key: one sparse, one consecutive.  
The first part comprises a spare sample of $\lambda/2$ messages selected 
according to a prescribed pattern in a first-in, first-out queue, 
over a specified section of the stream.  
For example, one might sample every 10,000th message.  
The section length should be chosen to exceed the time available to a transient attacker.
For example, at most 72,000 messages can be sent in 30~minutes,
even if messages are continuously sent using only one key.

The second part comprises the most recent $\lambda/2$ 
messages from the stream of valid messages received under the given group key.
This consecutive sample helps ensure changed state after a resynchronization.

\subsection{Key Establishment}
\label{key}

We briefly sketch a method for establishing the keys required by Mini-MAC.  Key management
is a vital and challenging task for vehicular security.  Currently, the CAN bus has no
provision for any cryptographic keys.

If the ECUs are capable of generating their own keys, 
we recommend that group keys be established as follows.  
Each ECU should generate its own key material,
which should never leave the ECU, except possibly in encrypted form.  Neither the ECU manufacturer, car manufacturer, 
nor car dealer should ever learn any keying material.  The security manager (perhaps a dealer) should be capable of initiating
a process for key establishment or rekeying.

Adding a Trusted Platform Module (TPM) to each ECU would provide a capability to
generate and store keys more securely.

To establish a group key, each group member should generate a key share of the same length as the group key.
Using a key-encrypting-key (KEK) shared by all ECUs in the group and used only for this purpose, 
each member sends its share to the other group members.  Each member computes the group key as the
XOR of the shares.  In the absense of public-key cryptography, we recommend that this KEK be
loaded into the ECUs during car manufacturing, using a trusted process that prevents anyone from
learning the KEK.

If the ECUs are incapable of generating their own keys, the following alternative procedure can
be used.  To facilitate the management of trust in establishing and updating keys, we recommend using a
hierarchy of keys, comprising key-group master keys, key-group KEKs, and ECU keys.
The manufacturer generates the master keys and loads them onto the ECUs.  
The car owner also possesses the master keys, which are
protected in a physical container (e.g., USB stick).  Using the master keys, 
and with the help of a trusted tool, the car owner can
change key-group KEKs via the OBD port, encrypting each new KEK with the corresponding master key
and loading it onto the appropriate ECUs.  By giving the security manager temporary KEKs, 
the car owner can grant temporary key management rights to the security manager.  In doing
so, the security manager can update the ECU keys and keyshares and perform maintenance
as needed.

Key management is a critical process that is potentially vulnerable to anyone with 
access to the ECUs or keying material.  In our suggested method, unencrypted keying material is never
transmitted on the CAN bus.



